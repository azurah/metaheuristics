\section{Formal Statement}

An Internet retail company wants to build several logistic centres in order to operate
in a new country. Its goal is to spend the minimum amount of money while making sure that
customers receive their products quickly enough.

\hfill

The company has a set of $\nloc$ locations $\loc = \{\eloc_1, \cdots, \eloc_\nloc\}$ where
logistic centres can be installed in, and a set of $\ncity$ cities $\city = \{\ecity_1,
\cdots, \ecity_\ncity\}$ that need to be served. For each location we know its coordinates
$u_\iloc = (u_{\iloc, x}, u_{\iloc, y}) \in \mathbb{R}^2$ ($1 \le \iloc \le \nloc$), and for
every city we know its coordinates $v_\icity = (v_{\icity, x}, v_{\icity, y}) \in
\mathbb{R}^2$ and its population $p_\icity \in \mathbb{N}$ ($1 \le \icity \le \ncity$). We
have available a set of $\ncentre$ logistic centre types $\centre = \{\ecentre_1, \cdots,
\ecentre_\ncentre\}$. Each type represents a logistic centre with working distance
$\omega_\icentre \in \mathbb{R}$, capacity $s_\icentre \in \mathbb{N}$ and installation cost
$i_\icentre \in \mathbb{R}$ ($1 \le \icentre \le \ncentre$).

\hfill

Not all logistic centres need to be installed, and not all locations have to have one centre
installed. However, if a centre is to be installed then it can only be so in at most one
location and in a way that there is at least a distance of $D$ to the rest of installed
centres. A centre may serve many cities - at least one if it is installed and none if it is
not - either as a primary or as a secondary centre, but it can never serve the same city as
both primary and secondary. Besides, centres must be within a certain range from the cities
they are serving. More precisely, a centre $\ecentre_\icentre \in \centre$ can only serve
city $\ecity_\icity \in \city$ as primary centre if the distance between the two is at most
$\omega_\icentre$ and as secondary centre if the city is at most at distance $3\omega_\icentre$.
Moreover, the sum of the population of those cities served by centre $\ecentre_\icentre$ as
primary centre plus 10\% of the sum of the population of those cities served by the same
centre as secondary centre can not exceed the capacity of the centre $\ecentre_\icentre$:
$s_\icentre$ ($1 \le \icity \le \ncity$, $1 \le \icentre \le \ncentre$). Finally, all cities
have to be served by exactly one primary and one secondary centre.

\hfill

The goal of the problem is to decide where to install logistic centres, determine of
which type each centre will be and what primary and secondary centres each city should
be connected to in order to minimise the total installation cost.



