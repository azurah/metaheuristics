\section{Formal Statement}
\label{sec:formal-statement}

An Internet retail company wants to build several logistic centres in order to operate
in a new country. Its goal is to spend the minimum amount of money while making sure that
customers receive their products quickly enough.

\hfill

The company has a set of $\nloc$ locations $\loc = \{\eloc_1, \cdots, \eloc_\nloc\}$ where
logistic centres can be installed in, and a set of $\ncity$ cities $\city = \{\ecity_1,
\cdots, \ecity_\ncity\}$ that need to be served. For each location we know its coordinates
$u_\iloc = (u_{\iloc, x}, u_{\iloc, y}) \in \mathbb{R}^2$ ($\forall \iloc : 1 \le \iloc \le
\nloc$), and for every city we know its coordinates $v_\icity = (v_{\icity, x}, v_{\icity, y})
\in \mathbb{R}^2$ and its population $p_\icity \in \mathbb{N}$ ($\forall \icity : 1 \le \icity
\le \ncity$). We have available a set of $\ncentre$ logistic centre types $\centre =
\{\ecentre_1, \cdots, \ecentre_\ncentre\}$. Each type is characterised by a working distance
$\omega_\icentre \in \mathbb{R}$, capacity $s_\icentre \in \mathbb{N}$ and installation cost
$i_\icentre \in \mathbb{R}$ ($\forall \icentre : 1 \le \icentre \le \ncentre$).

\hfill

The goal is to know the type of the centres to be installed in the different locations in
order to serve all the different cities as quickly and as cheaply as possible. Not all logistic
centres types need to be used, and not all locations have to have one centre installed. If a
location requires a centre to be installed in it then it can be at most one centre of one type
but many centres of the same type can be installed in more than one location (that is,
different locations may have a centre of the same type installed in them). However, centres can
only be installed in those locations that are at least at a distance of $D$ to the rest of
locations that have a centre installed in them.

\hfill

The installed centre of type $\ecentre_\icentre$ in a location $\eloc_\iloc$ may be acting as a
primary or as a secondary centre when serving the different cities $\ecity_\icity$, and not
necessarily as primary or as secondary all the time: it may serve as a primary centre to some
of the cities and as a secondary centre to the rest. All cities need to be served exactly by
a primary centre and a secondary centre. From now on, when a location $\eloc_\iloc$ has a
centre of type $\ecentre_\icentre$ installed in it that serves some cities as a primary centre
and some other cities as a secondary centre we will say that that location is serving those
cities with a primary and with a secondary role, respectively. Now, a location may serve many
cities - at least one if it has a centre installed and none if no centre is installed in it -
but it can never serve the same city with a primary and a secondary role at the same time.
Besides, a location $\eloc_\iloc$ with a centre of type $\ecentre_\icentre$ installed in it
can only serve cities $\ecity_\icity$ with a primary role if the distance between the cities
and the location is less than the working distance of the centre installed: $d(\eloc_\iloc,
\ecity_\icity) \le \omega_\icentre$. But, if the location is serving cities with a secondary
role then the distance between the location and the cities can be at most three times the
working distance of that centre: $d(\eloc_\iloc, \ecity_\icity) \le 3\omega_\icentre$. One last
constraint is that the sum of the population of those cities $\ecity_\icity$ served by a
location $\eloc_\iloc$ (that has a centre of type $\ecentre_\icentre$ installed in it) with a
primary role plus 10\% of the sum of the population of those cities served by the same
location with a secondary role can not exceed the capacity of the centre $s_\icentre$.

